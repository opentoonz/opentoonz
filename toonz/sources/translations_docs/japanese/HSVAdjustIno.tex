\documentclass[a4paper,12pt]{article}
\usepackage[a4paper, total={180mm, 272mm}]{geometry}

\usepackage{fontspec}
\setmainfont[Path=fonts/, Extension=.ttf]{ipaexm}

\setlength\parindent{3.5em}
\setlength\parskip{0em}
\renewcommand{\baselinestretch}{1.247}

\begin{document}

\thispagestyle{empty}

\Large
\noindent \\
HSV Adjust Ino\medskip
\par
\normalsize
色相(H)、彩度(S)、明度(V)、に scale をかけ、次に、shift をします。\\
\\
-{-}- \ 入力 \ -{-}-\\
Source\par
処理をする画像を接続します。\\
Reference\par
Pixel 毎に効果の強弱をつけるための参照画像を接続します。\\
\\
-{-}- \ 設定 \ -{-}-\\
Pivot\par
\noindent \ \ \, Scale をかける時の中心値を指定します。\\
\par
\noindent \ \ \, Hue\par
色相(hue)について Scale の中心値を指定します。\par
最小は0.0、最大は360.0です。\par
初期値は0.0です。\par
\noindent \ \ \, Saturation\par
彩度(saturation)について Scale の中心値を指定します。\par
最小は0.0、最大は1.0です。\par
初期値は0.0です。\par
\noindent \ \ \, Value\par
明度(brightness Value)について Scale の中心値を指定します。\par
最小は0.0、最大は1.0です。\par
初期値は0.0です。\\
\\
Scale\par
\noindent \ \ \, Pivot 値を中心として Scale をかけて HSV の範囲を拡大あるいは縮小します。\par
\noindent \ \ \, Hue 値は円の上を回るように再帰しますが、\par
\noindent \ \ \, Saturation,Value 値はゼロ以上はゼロ、1以上は1で固定します。\\
\par
\noindent \ \ \, Hue\par
色相(hue)について Scale をかけます。\par
最小は0.0です。\par
初期値は1.0です。\par
\noindent \ \ \, Saturation\par
彩度(saturation)について Scale をかけます。\par
最小は0.0です。\par
初期値は1.0です。\par
\noindent \ \ \, Value\par
明度(brightness Value)について Scale をかけます。

\newpage

\thispagestyle{empty}

\ \vspace{-0.2em}
\par
最小は0.0です。\par
初期値は1.0です。\\
\\
Shift\par
\noindent \ \ \, Shift して HSV の値をずらします。\par
\noindent \ \ \, Hue 値は円の上を回るように再帰しますが、\par
\noindent \ \ \, Saturation,Value 値はゼロ以上はゼロ、1以上は1で固定します。\\
\par
\noindent \ \ \, Hue\par
色相(hue)について Shift します。\par
初期値は0.0です。\par
\noindent \ \ \, Saturation\par
彩度(saturation)について Shift します。\par
初期値は0.0です。\par
\noindent \ \ \, Value\par
明度(brightness Value)について Shift します。\par
初期値は0.0です。\\
\\
Premultiplied\par
ON なら、RGB に対して Premultiply 済の\par
(Alpha チャンネルの値があらかじめ RGB チャンネルに乗算されている)\par
画像として処理します。\par
初期値は ON です。\\
\\
Reference\par
Pixel 毎に効果の強弱をつけるための参照画像の値の取り方を選択します。\par
入力の\textquotedbl Reference\textquotedbl に画像を接続し、\par
Red/Green/Blue/Alpha/Luminance/Nothing から選びます。\par
この効果をつけたくないときは Nothing を選ぶか、接続を切ります。\par
初期値は Red です。

\end{document}