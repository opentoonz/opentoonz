\documentclass[a4paper,12pt]{article}
\usepackage[a4paper, total={180mm, 272mm}]{geometry}

\usepackage{fontspec}
\setmainfont[Path=fonts/, Extension=.ttf]{ipaexm}

\setlength\parindent{3.5em}
\setlength\parskip{0em}
\renewcommand{\baselinestretch}{1.247}

\begin{document}

\thispagestyle{empty}

\Large
\noindent \\
PN Clouds Ino\medskip
\par
\normalsize
雲模様の絵を生成します。\\
\par
pixel 値は、\par
\noindent \hskip 7em \ \ 8bits 画像の場合128(0x80)\par
\noindent \hskip 7em 16bits 画像の場合32768(0x8000)\par
を中心としたノイズとなります。\par
なお、この値より大きい値と小さい値は\par
必ずしも均等にはなりません。\\
\\
-{-}- \ 設定 \ -{-}-\\
Size\par
模様のパターンの大きさを指定します。\par
値を小さくすると小さい模様、大きくすると大きい模様となります。\par
単位は mm です。\par
初期値は10です。\\
\\
Z\par
絵を連続的に変化させます。\par
例えば、1フレームから24フレームを0から1まで変化させます。\par
初期値は0です。\\
\\
Octaves\par
雲模様の細部を加えます。\par
1から10までの整数で指定します。\par
数が増えるほど細部のノイズが現れます。\par
初期値は1です。\\
\\
Persistance\par
雲模様の細部のノイズの強さを指定します。\\
\\
Alpha Rendering\par
OFFのとき Alphaを最大値で塗りつぶし、全面不透明とします。\par
ONで Alphaにも RGB と同じ画像を生成します。\par
初期値は ONです。

\end{document}