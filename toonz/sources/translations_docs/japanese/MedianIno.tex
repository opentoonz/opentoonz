\documentclass[a4paper,12pt]{article}
\usepackage[a4paper, total={180mm, 272mm}]{geometry}

\usepackage{fontspec}
\setmainfont[Path=fonts/, Extension=.ttf]{ipaexm}

\setlength\parindent{3.5em}
\setlength\parskip{0em}
\renewcommand{\baselinestretch}{1.247}

\begin{document}

\thispagestyle{empty}

\Large
\noindent \\
Median Ino\medskip
\par
\normalsize
ノイズを低減し、多数派あるいは中間の色で侵食し、絵の輪郭を丸めます\\
\\
-{-}- \ 入力 \ -{-}-\\
Source\par
処理をする画像を接続します。\\
Reference\par
Pixel 毎に効果の強弱をつけるための参照画像を接続します。\\
\\
-{-}- \ 設定 \ -{-}-\\
Radius\par
侵食する範囲を円の半径で指定します。\par
単位は mm です。\\
\par
ゼロ以上の値で指定します。最大は100mm です。\par
ピクセルの幅より小さいと(まわりの Pixel を含めないため)\par
なにもしません。\\
\par
初期値は0.35mm です。\\
\\
Channel\par
medianをかける色チャンネルを指定します。\\
\par
\textquotedbl Red\textquotedbl\par
\textquotedbl Green\textquotedbl\par
\textquotedbl Blue\textquotedbl\par
\textquotedbl Alpha\textquotedbl\par
を選ぶと、指定の色チャンネルに処理をかけて、\par
その結果を RGBAチャンネルに保存します。\par
白黒画像では、このように単チャンネル処理をすることで、\par
処理スピードがはやくなります。\\
\par
\textquotedbl All\textquotedbl\par
では、RGBA各チャンネルに処理をかけます。\\
\par
初期値は\textquotedbl All\textquotedbl です。\\
\\
Reference\par
Pixel 毎に効果の強弱をつけるための参照画像の値の取り方を選択します。\par
入力の\textquotedbl Reference\textquotedbl に画像を接続し、\par
Red/Green/Blue/Alpha/Luminance/Nothingから選びます。

\newpage

\thispagestyle{empty}

\ \vspace{-0.2em}
\par
この効果をつけたくないときは Nothingを選ぶか、接続を切ります。\par
初期値は Red です。

\end{document}