\documentclass[a4paper,12pt]{article}
\usepackage[a4paper, total={180mm, 272mm}]{geometry}

\usepackage{fontspec}
\setmainfont[Path=fonts/, Extension=.ttf]{ipaexm}

\setlength\parindent{3.5em}
\setlength\parskip{0em}
\renewcommand{\baselinestretch}{1.247}

\begin{document}

\thispagestyle{empty}

\Large
\noindent \\
Channel Selector Ino\medskip
\par
\normalsize
複数の入力(Source1,2,3,4)の RGBAから各チャンネルを選びます。\\
\\
-{-}- \ 入力 \ -{-}-\\
Source1\\
Source2\\
Source3\\
Source4\par
処理をする画像を一つ以上任意に接続します。\par
最大四つまでつなげます。\\
\\
-{-}- \ 選択 \ -{-}-\\
Red\par
\noindent \ \, Image\par
赤チャンネルとして使う入力画像を指定します。\par
Source1/Source2/Source3/Source4の中から選びます。\par
1から4までの値をキーボード入力します。\par
初期値は0です。\\
\par
\noindent \ \, Channel\par
赤チャンネルとして使うチャンネルを、\par
Red/Green/Blue/Alpha から選びます。\par
初期値は Red です。\\
\\
Green\par
\noindent \ \, Image\par
緑チャンネルとして使う入力画像を指定します。\par
その他は\textquotedbl Red Image\textquotedbl と同じです。\\
\par
\noindent \ \, Channel\par
緑チャンネルとして使うチャンネルを、\par
Red/Green/Blue/Alpha から選びます。\par
初期値は Greenです。\\
\\
Blue\par
\noindent \ \, Image\par
青チャンネルとして使う入力画像を指定します。\par
その他は\textquotedbl Red Image\textquotedbl と同じです。\\
\par
\noindent \ \, Channel\par
青チャンネルとして使うチャンネルを、

\newpage

\thispagestyle{empty}

\ \vspace{-0.2em}
\par
Red/Green/Blue/Alpha から選びます。\par
初期値は Blueです。\\
\\
Alpha\par
\noindent \ \, Image\par
アルファチャンネルとして使う入力画像を指定します。\par
その他は\textquotedbl Red Image\textquotedbl と同じです。\\
\par
\noindent \ \, Channel\par
アルファチャンネルとして使うチャンネルを、\par
Red/Green/Blue/Alpha から選びます。\par
初期値は Alphaです。

\end{document}