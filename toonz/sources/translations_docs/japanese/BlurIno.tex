\documentclass[a4paper,12pt]{article}
\usepackage[a4paper, total={180mm, 272mm}]{geometry}

\usepackage{fontspec}
\setmainfont[Path=fonts/, Extension=.ttf]{ipaexm}

\setlength\parindent{3.5em}
\setlength\parskip{0em}
\renewcommand{\baselinestretch}{1.247}

\begin{document}

\thispagestyle{empty}

\Large
\noindent \\
Blur Ino\medskip
\par
\normalsize
ぼかします。\\
\par
手法として、ガウス分布を使います。\par
そのため、結果画像に指向性はありません。\\
\par
Alphaチャンネルも処理します。\\
\\
-{-}- \ 入力 \ -{-}-\\
Source\par
処理をする画像を接続します。\\
\\
Reference\par
Pixel 毎に効果の強弱をつけるための参照画像を接続します。\\
\\
-{-}- \ 設定 \ -{-}-\\
Radius\par
ぼかしの半径を指定します。\par
単位は mm です。\\
\\
Reference\par
Pixel 毎に効果の強弱をつけるための参照画像の値の取り方を選択します。\par
入力の\textquotedbl Reference\textquotedbl に画像を接続し、\par
Red/Green/Blue/Alpha/Luminance/Nothingから選びます。\par
この効果をつけたくないときは Nothingを選ぶか、接続を切ります。\par
初期値は Red です。

\end{document}