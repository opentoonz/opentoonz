\documentclass[a4paper,12pt]{article}
\usepackage[a4paper, total={180mm, 272mm}]{geometry}

\usepackage{fontspec}
\setmainfont[Path=fonts/, Extension=.ttf]{ipaexm}

\setlength\parindent{3.5em}
\setlength\parskip{0em}
\renewcommand{\baselinestretch}{1.247}

\begin{document}

\thispagestyle{empty}

\Large
\noindent \\
Add Ino\medskip
\par
\normalsize
単純加算\par
足して明るくします\par
式 = Back + Fore\\
\\
-{-}- \ 入力 \ -{-}-\par
両方とも接続していれば合成処理します。\par
動作スイッチ OFF のとき Back を表示します。\par
片方のみ繋がっていればそれを表示します。\\
Fore\par
上に重ねる画像を接続します。\\
Back\par
下に置く画像を接続します。\\
\\
-{-}- \ 設定 \ -{-}-\\
Opacity\par
上に重ねる画像の不透明度を指定します。\par
0の時は Fore 画像は透明になります。\par
初期値は\textquotedbl1.0\textquotedbl で Fore 画像は不透明として合成します。\par
0から1.0の間の値を指定します。\par
1以上の10.0までの値も指定できます\\
\\
Clipping Mask\par
ON にすると、\par
素材(Back)の存在しない(Alpha 値がゼロ)場所は、透明のままにします。\par
初期設定は ON です。

\end{document}