\documentclass[a4paper,12pt]{article}
\usepackage[a4paper, total={180mm, 272mm}]{geometry}

\usepackage{fontspec}
\setmainfont[Path=fonts/, Extension=.ttf]{ipaexm}

\setlength\parindent{3.5em}
\setlength\parskip{0em}
\renewcommand{\baselinestretch}{1.247}

\begin{document}

\thispagestyle{empty}

\Large
\noindent \\
Fog Ino\medskip
\par
\normalsize
光を散らします。\\
\par
水の中や、風呂の中、霧の中などで、\par
光が散乱(Light Scattering)して見える効果をねらって作成しました。\par
光学フィルターにおける、フォグや、ディフュージョンのような\par
効果をねらっていますが、光をシミュレートしているわけではありません。\\
\par
各 pixel において、周りのより明るい pixelから、光るさの影響を受けます。\par
近くの pixel からはより強く、遠くの pixelからは弱い影響になります。\\
\par
初めに、\textquotedbl Alpha Rendering\textquotedbl が ONなら Alpha チャンネルを処理し、\par
次に、Alphaチャンネルがゼロでないピクセルの RGBを処理します。\\
\par
\textquotedbl Alpha Rendering\textquotedbl が OFFなら Alphaチャンネルを処理せず、\par
RGB画像の変化に Mask しないので、マスクエッジにジャギーが出ます。\\
\\
-{-}- \ 入力 \ -{-}-\\
Source\par
処理をする画像を接続します。\\
\\
-{-}- \ 設定 \ -{-}-\\
Radius\par
光を散らす範囲を、円半径で指定します。\par
単位はミリメートルです。\par
ゼロ以上の値で指定します。最大は100です。\par
Pixel 幅より小さい値の時は光が散らず、Fogはかかりません。\par
半径は大きくするほど処理に時間がかかります。\par
初期値は1です。\\
\\
Curve\par
散っていく光の減衰カーブです。\par
0.01以上の値で指定します。最大は100です。\par
離れた pixel ほどその影響は弱くなりますが、\par
その変化を Gamma曲線で表わします。\par
1.0の場合は明るさはリニアに減衰します。\par
値が小さいほど明るさが細り(影響が急に減り)、\par
大きいほど明るさが膨らみ(影響が強調され)ます。\par
初期値は1です。

\newpage

\thispagestyle{empty}

\ \vspace{-0.2em}
\par
\noindent Power\par
光りを散らす強さを変えます。\par
0から1の範囲で指定します。\par
1.0が最大値ですがさらに2.0まで指定できて、光を強調できます。\par
強調では墨線等の暗い部分が発光することがあります。\par
0.0の時は光が散らず、Fogはかかりません。\par
0.0より小さい値を-2.0まで与えることができます。\par
この時の処理は光の散乱ではなく、暗さの散乱となります。\par
初期値は1です。\\
\\
Threshold Min\\
Threshold Max\par
この値を含めてそれ以上の明るさの Pixel が光を放ちます。\\
\par
より明るい Pixel から明るさの影響を受けますが、それに加え、\par
値(\textquotedbl Threshold Min\textquotedbl )以上の輝度の Pixel であったときに、\par
明るさの影響を受けます。\par
輝度は、Pixel 値の RGBから求めた(HLS の)L値です。\\
\par
0.0以上1.01以下の範囲で指定します。\\
\par
両方とも1.01とすると Fogはかかりません。\\
\par
\textquotedbl Threshold Max\textquotedbl が\textquotedbl Threshold Min\textquotedbl より大きい場合、\par
Min から Max の間をリニア補間して fogの変化をなめらかに行ないます。\\
\par
\textquotedbl Threshold Max\textquotedbl をゼロとしておく(Min より小さければよい)ことで、\par
\textquotedbl Threshold Min\textquotedbl 以上の明るさからいきなり光を放ちます。\par
\textquotedbl Threshold Min\textquotedbl もゼロにすれば全体に Fogがかかります。\\
\par
初期値はどちらも0です。\\
\\
Alpha Rendering\par
Alphaチャンネルがあるときのみ有効なスイッチです。\par
OFFのときはなにもしません。\par
ONで Alphaにも処理をします。\par
初期値は OFFです。

\end{document}