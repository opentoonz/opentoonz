\documentclass[a4paper,12pt]{article}
\usepackage[a4paper, total={180mm, 272mm}]{geometry}

\usepackage{fontspec}
\setmainfont[Path=fonts/, Extension=.ttf]{ipaexm}

\setlength\parindent{3.5em}
\setlength\parskip{0em}
\renewcommand{\baselinestretch}{1.247}

\begin{document}

\thispagestyle{empty}

\Large
\noindent \\
Max Min Ino\medskip
\par
\normalsize
It brightens (dark) portions of the image.\\
\par
It inflates around (the polygon).\par
In addition, the change is smooth.\\
\par
First, it processes the Alpha channel, if specified.\par
Then, it handles the Pixel RGB if the Alpha channel is not zero.\\
\\
-{-}- \ Inputs \ -{-}-\\
Source\par
Connect the image to process.\\
Reference\par
Connect the reference image to put the strength of the inflate effect for each Pixel.\\
\\
-{-}- \ Settings \ -{-}-\\
Max Min Select\par
Specify how to handle the selection.\par
\textquotedbl Max\textquotedbl \ -> Inflate the bright parts of the image\par
\textquotedbl Min\textquotedbl \ -> Inflate the dark parts of the image\par
Using \textquotedbl Min\textquotedbl , ink lines of the cell image outline, the transmission region of the outside\par
will be painted zero, ink line bulge of the transparent area will continue to disappear.\par
Also it increases the transmission region Alpha as well.\par
The default setting is \textquotedbl Max\textquotedbl .\\
\\
Radius\par
The inflate size, specified in the circle radius.\par
The unit is millimeters.\par
Specify a number greater than or equal to 0.\par
By adding Smoothing (in pixels) it does not inflate using a value less than 1.\par
Therefore, if the value is less, there will be an effect with a fine image,\par
but it may not take effect on a rough image.\par
A larger Radius will take more time to process.\\
\\
Polygon Number\par
Other than inflating to a circle, you can specify whether to inflate to a polygon.\par
Specify an integer value.\par
A value of 2 will inflate around the circle Radius.\par
3 or more, will inflate to the number of angles of the polygon. The maximum is 16.\par
It is a polygon that begins from the true right of the center to inflate.\par
The default value is 2.

\newpage

\thispagestyle{empty}

\ \vspace{-0.2em}
\\
\par
\noindent Degree\par
\textquotedbl Polygon Number\textquotedbl \ value of 3 or more, specify the slope of the inflate polygon.\par
\textquotedbl Polygon Number\textquotedbl \ value of 2 or less will have no effect.\par
Specify in Degree units more than 0.\par
It will rotate in a clockwise direction.\par
The default value is 0.\\
\\
Alpha Rendering\par
When ON it will also process the Alpha channel.\par
When OFF it will process only to RGB. It uses the Alpha channel without BG image.\par
The default setting is ON.\\
\\
Reference\par
Choose how Reference image values put the strength of the effect into each Pixel.\par
An image is connected to the \textquotedbl Reference\textquotedbl \ of the input,\par
Choose from Red/Green/Blue/Alpha/Luminance/Nothing.\par
Choose Nothing when you do not want this effect, it will turn off the connection.\par
The default setting is Red.

\end{document}