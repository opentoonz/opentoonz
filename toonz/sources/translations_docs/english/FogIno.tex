\documentclass[a4paper,12pt]{article}
\usepackage[a4paper, total={180mm, 272mm}]{geometry}

\usepackage{fontspec}
\setmainfont[Path=fonts/, Extension=.ttf]{ipaexm}

\setlength\parindent{3.5em}
\setlength\parskip{0em}
\renewcommand{\baselinestretch}{1.247}

\begin{document}

\thispagestyle{empty}

\Large
\noindent \\
Fog Ino\medskip
\par
\normalsize
It scatters light.\\
\par
Used for bath water, and mist, etc.,\par
A directional Light creates the appearance of the (Light Scattering) effect.\par
The optical filter, and fog, such as diffusion\par
can be aimed at by the effect, but it does not have to simulate the light.\\
\par
Each pixel, is affected by the shining of a more bright surrounding pixel.\par
More strongly from nearby pixels, it will have a weaker impact from distant pixels.\\
\par
First, \textquotedbl Alpha Rendering\textquotedbl \ processing starts if the Alpha channel is ON,\par
next, it handles the RGB pixels when the Alpha channel is not zero.\\
\par
\textquotedbl Alpha Rendering\textquotedbl \ is not handled when the Alpha channel is OFF,\par
it does not Mask changes in the RGB image, so you will get jaggies on the mask edge.\\
\\
-{-}- \ Inputs \ -{-}-\\
Source\par
Connect the image to process.\\
\\
-{-}- \ Settings \ -{-}-\\
Radius\par
The extent of which to scatter the light, specified by a circle radius.\par
The unit is millimeters.\par
Specify a value greater than or equal to 0. The maximum is 100.\par
When a Pixel is smaller than the width value, it is not affected, Fog will not be applied.\par
A larger Radius will take more time to process.\par
The default value is 1.\\
\\
Curve\par
The attenuation curve going towards the scattered light.\par
Specify a value of 0.01 or more. The maximum is 100.\par
The effect will be weaker on pixels further away,\par
it represents the change in the Gamma curve.\par
Brightness in the case of 1.0 is linear attenuated.\par
The smaller the value the lower the brightness becomes (impact is reduced abruptly),\par
a larger bulge will produce a higher brightness (impact is emphasized).\par
The default value is 1.

\newpage

\thispagestyle{empty}

\ \vspace{-0.2em}
\par
\noindent Power\par
Changes the strength to disperse the light.\par
Specify a value in the range 0 to 1.\par
You can specify values above 1.0 up to a maximum 2.0, it will emphasize the light.\par
The emphasis may be useful to emit light from a dark portion such as an ink line.\par
When the value is 0.0 pixels will not be affected by the light, Fog will not be applied.\par
You can specify values less than 0.0 down to a minimum of -2.0.\par
Processing with this is not the scattering of light, it will be the scattering of darkness.\par
The default value is 1.\\
\\
Threshold Min\\
Threshold Max\par
Specify these values to get more brightness from the emitted Pixel light.\\
\par
Affected by the brightness from a more bright Pixel, but in addition to that,\par
using the Pixel value (\textquotedbl Threshold Min\textquotedbl ) can get even more brightness,\par
to affect the current overall brightness.\par
Brightness, is determined from the RGB Pixel value (of HLS) from the L value.\\
\par
The following range greater than or equal to 0.0 to 1.01 can be specified as values.\\
\par
If both values are 1.01 Fog will not be applied.\\
\par
If \textquotedbl Threshold Max\textquotedbl \ is greater than \textquotedbl Threshold Min\textquotedbl ,\par
it carries out smooth changes in fog by linear interpolation between Min and Max.\\
\par
Reversing \textquotedbl Threshold Max\textquotedbl \ (which may be less than Min) but kept above 0,\par
will suddenly emit light from pixels with \textquotedbl Threshold Min\textquotedbl \ or more of brightness.\par
\textquotedbl Threshold Min\textquotedbl \ also produces full Fog if set to 0.\\
\par
The default value is 0 for both.\\
\\
Alpha Rendering\par
Switch is only valid when there is an Alpha channel in the image.\par
When OFF it does not do anything.\par
When ON it will also process the Alpha channel.\par
The default setting is OFF.

\end{document}